% Complete recipe example
\begin{recipe}
[% 
    preparationtime = {\unit[1]{h}},
    bakingtime={\unit[12 - 15]{min}},
    bakingtemperature={\protect\bakingtemperature{
        fanoven=\unit[375]{°F}}},
        %topbottomheat=\unit[195]{°C},
        %topheat=\unit[195]{°C},
        %gasstove=Level 2}},
    portion = {\portion{16}},
    %calory={\unit[3]{kJ}},
    %source = {Somebody you used know \url{www.}}
]
{Chocolate Roll}
    
    \graph
    {% pictures
        small=pic/glass,     % small picture
        big=pic/ingredients  % big picture
    }
    
    %\introduction{

    %}
    
    \ingredients{%
        3 & Eggs \\
        1 cup & Sugar \\
        5 Tbls & Water \\
        1 tsp & Vanilla \\
        1 cup & Flour \\
        1 tsp & Baking Powder \\
        1/2 tsp & Baking Soda \\
        1 1/2 Tbls & Cocoa \\
        1 cup & Whipping Cream \\
        1/4 cup & Sugar \\
    }
    
    \preparation{%
        \step Beat egg well.
        \step Alternate additions of water and sugar to the egg.
        \step Sift flour, baking powder, baking soda, and cocoa.
        \step Incorporate the sifted dry ingredients to the egg mixture.
        \step Pour onto a carefully greased pan that is lined with parchment.
        \step Bake, and immediately upon removal from oven invert onto a tea towel covered in powdered sugar and roll.
        \step Periodically unroll the cake to allow steam to escape.
        \step Beat sugar and whipping cream until desired consistency and spread.
    }
    
    %\suggestion[Finishing]
    %
    
    %\suggestion
    
    %\hint{%
   %     Enjoy typesetting recipes with {\textbf{\Large\LaTeX}} and {\textbf{\Large xcookybooky!}}
  %  }
    
\end{recipe}