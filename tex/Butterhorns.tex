% Complete recipe example
\begin{recipe}
[% 
    preparationtime = {\unit[4]{h}},
    bakingtime={\unit[12-15]{min}},
    bakingtemperature={\protect\bakingtemperature{
        fanoven=\unit[375]{°F}}},
        %topbottomheat=\unit[195]{°C},
        %topheat=\unit[195]{°C},
        %gasstove=Level 2}},
    portion = {\portion{32}},
    %calory={\unit[3]{kJ}},
    source = {Mrs. Drake}
]
{Butterhorns}
    
    \graph
    {% pictures
        small=pic/glass,     % small picture
        big=pic/ingredients  % big picture
    }
    
    %\introduction{

    %}
    
    \ingredients{%
        5 tsp & Yeast \\
        1/3 cup & Warm Water \\
        2 cups & Warm Milk \\
        9 cups & Flour \\
        1 cup & Shortening \\
        1 cup & Sugar \\
        6 & Eggs \\
        2 tsp & Salt \\
        4 Tbls & Butter \\
    }
    
    \preparation{%
        \step Dissolve yeast in water.  Add 4 cups of flour, milk, shortening, sugar, eggs,m and salt.  Beat for 2 minutes.  Add remaining flour to make dough soft.
        \step Knead lightly and place in greased bowl to raise for 2 - 3 hours.
        \step Punch down and divide into 4 parts.  Roll each into a 9 inch circle.  Brush with butter and cut into 8 pie-like wedges.  Roll each wedge from the edge down.
        \step Freeze.  Thaw for 5 hours and bake.
    }
    
    %\suggestion[Finishing]
    %
    
    \suggestion{%
		If you don't desire to freeze some or all of the rolls, let rise again instead of freezing.
    }
    
    %\hint{%
   %     Enjoy typesetting recipes with {\textbf{\Large\LaTeX}} and {\textbf{\Large xcookybooky!}}
  %  }
    
\end{recipe}