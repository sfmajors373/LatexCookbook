%Complete recipe example
\begin{recipe}
[% 
    preparationtime = {\unit[1]{h}},
    %bakingtime={\unit[12-15]{min}},
    %bakingtemperature={\protect\bakingtemperature{
        %fanoven=\unit[375]{°F}}},
        %topbottomheat=\unit[195]{°C},
        %topheat=\unit[195]{°C},
        %gasstove=Level 2}},
    portion = {\portion{4}},
    %calory={\unit[3]{kJ}},
    source = {Pampered Chef}
]
{Glazed Pork Chops with Grilled Apples}
    
    \graph
    {% pictures
        small=pic/glass,     % small picture
        big=pic/ingredients  % big picture
    }
    
    %\introduction{

    %}
    
    \ingredients{%
        4 & pork chops \\
	1 tsp & Dried Rosemary \\
	1 clove & Garlic, pressed \\
	1/2 tsp & Salt \\
	1/4 tsp & Ground Pepper \\
	2 tsp & Gingerroot, finely chopped \\
	1/3 cup & Apricot Preserves \\
	3 Tbls & Dijon mustard \\
	2 & Apples, cored and cut crosswise \\
    }
    
    \preparation{%
        \step Mix rosemary, garlic, salt and pepper in small bowl.
        \step Prepare grill for direct cooking over medium coals.  Season both sides of pork chops with 1 tsp of seasoning mixture.
	\step Mix gingerroot, preserves, mustard and remaining seasoning mix.
	\step Lightly grease grid of grill.  Brush one side of apples and pork chops with glaze.  Place glazed side down on grill.
	\step Grill apples 3 minutes and pork chops 5 minutes, covered turning once.  Reglaze, flip, cooking 3-5 minutes.
    }
    
    %\suggestion[Finishing]
    %
    
    \suggestion{%
	Braeburn or Gala apples are recommended.  Though, the pork chops are good even without the apples.  Also, this can be made in a pan instead of on a grill.
    }
    
    %\hint{%
   %     Enjoy typesetting recipes with {\textbf{\Large\LaTeX}} and {\textbf{\Large xcookybooky!}}
  %  }
    
\end{recipe}
