% Complete recipe example
\begin{recipe}
[% 
    preparationtime = {\unit[1]{h}},
    bakingtime={\unit[10 - 12]{min}},
    bakingtemperature={\protect\bakingtemperature{
        fanoven=\unit[350]{°F}}},
        %topbottomheat=\unit[195]{°C},
        %topheat=\unit[195]{°C},
        %gasstove=Level 2}},
    portion = {\portion{16}},
    %calory={\unit[3]{kJ}},
    %source = {Somebody you used know \url{www.}}
]
{Chocolate Cake}
    
    \graph
    {% pictures
        small=pic/glass,     % small picture
        big=pic/ingredients  % big picture
    }
    
    %\introduction{

    %}
    
    \ingredients{%
        1 cup & Shortening \\
        1 cup & Peanut Butter \\
        1 1/2 cups & Sugar \\
        1/2 cup & Brown Sugar \\
        1 1/2 tsp & Baking Soda \\
        2 & Eggs \\
        1 tsp & Vanilla \\
        2 1/4 cups & Flour \\
    }
    
    \preparation{%
        \step Mix shortening and peanut butter in large mixing bowl.
        \step Add sugar, brown sugar, and baking soda.
        \step Incorporate eggs and vanilla.
        \step Slowly add flour.
        \step Form balls of dough, roll in sugar and press with fork to make a cross hatch design on top.
        \step Bake for 10 - 12 minutes.
    }
    
    \suggestion[Finishing]
    {%
    If you are feeling whimsical, roll the dough into multiple different sized balls for a head, body, arms, legs, ears, and nose and assemble into tiny teddy bears with mini chocolate chip eyes.    
    }
    
    %\suggestion
    
    %\hint{%
   %     Enjoy typesetting recipes with {\textbf{\Large\LaTeX}} and {\textbf{\Large xcookybooky!}}
  %  }
    
\end{recipe}