% Complete recipe example
\begin{recipe}
[% 
    preparationtime = {\unit[1 1/2]{h}},
    bakingtime={\unit[15 ]{min}},
    bakingtemperature={\protect\bakingtemperature{
        fanoven=\unit[350]{°F}}},
        %topbottomheat=\unit[195]{°C},
        %topheat=\unit[195]{°C},
        %gasstove=Level 2}},
    portion = {\portion{16}},
    %calory={\unit[3]{kJ}},
    %source = {Somebody you used know \url{www.}}
]
{Pepperoni Rolls}
    
    \graph
    {% pictures
        small=pic/glass,     % small picture
        big=pic/ingredients  % big picture
    }
    
    %\introduction{

    %}
    
    \ingredients{%
        1 1/2 cups & Water \\
        1 tsp & Yeast \\
        2 Tbls & Sugar \\
        1 tsp & Parmesan \\
        1 & Egg \\
        1/3 cup & Oil \\
        4 - 5 cups & Flour \\
    }
    
    \preparation{%
        \step Mix all ingredients.  Knead until smooth.
        \step Let rise until doubled.
        \step Divide in half.  Roll into a rectangle and cut into 8 pieces.  Add fillings to each little rectangle. Roll it up and place on baking stone/sheet.
        \step Bake until golden.
    }
    
    %\suggestion[Finishing]
    %
    
    \suggestion{%
		Don't roll the dough too thin - they won't turn out quite as well.
    }
    
    %\hint{%
   %     Enjoy typesetting recipes with {\textbf{\Large\LaTeX}} and {\textbf{\Large xcookybooky!}}
  %  }
    
\end{recipe}\textbf{}