% Complete recipe example
\begin{recipe}
[% 
    %preparationtime = {\unit[1/2]{h}},
    bakingtime={\unit[15]{min}},
    bakingtemperature={\protect\bakingtemperature{
        fanoven=\unit[375]{°F}}},
        %topbottomheat=\unit[195]{°C},
        %topheat=\unit[195]{°C},
        %gasstove=Level 2}},
    %portion = {\portion{16}},
    %calory={\unit[3]{kJ}},
    %source = {Somebody you used know \url{www.}}
]
{Pumpkin Roll}
    
    \graph
    {% pictures
        small=pic/glass,     % small picture
        big=pic/ingredients  % big picture
    }
    
    %\introduction{

    %}
    
    \ingredients{%
        3 & Eggs \\
        1 cup & Sugar\\
        2/3 cup + 1 tsp & Pumpkin\\
        1 tsp & Lemon Juice\\
        3/4 cup & Flour\\
        2 tsp & Cinnamon\\
        1 tsp & Baking Powder\\
        Filling:\\
        1 1/2 cups & Powdered Sugar\\
        6 oz & Cream Cheese\\
        1/4 cup & Butter\\
        1/2 tsp & Vanilla\\
    }
    
    \preparation{%
        \step Beat eggs on high speed for 5 minutes.
        \step Add sugar, pumpkin and lemon juice.
        \step Gradually add dry ingredients.
        \step Pour into jelly roll pan that has been greased and lined with greased parchment.
        \step Bake for 15 minutes.
        \step Remove from pan and lay onto tea towel coated with powdered sugar.  Roll from short end.
        \step Periodically unroll gently to release steam until it is cool.
        \step Mix all filling ingredients in mixer.
        \step Unroll cake from towel and gently spread filling.

    }
    
    %\suggestion[Finishing]
    %
    
    \suggestion{%
    If using a large can of pumpkin, it will make four rolls.  They can be rolled in wax paper, saran, and foil and frozen until desired.
    }
    
    %\hint{%
   %     Enjoy typesetting recipes with {\textbf{\Large\LaTeX}} and {\textbf{\Large xcookybooky!}}
  %  }
    
\end{recipe}