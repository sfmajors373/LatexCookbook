% Complete recipe example
\begin{recipe}
[% 
    preparationtime = {\unit[1 1/2]{h}},
    bakingtime={\unit[1]{h}},
    bakingtemperature={\protect\bakingtemperature{
        fanoven=\unit[350]{°F}}},
        %topbottomheat=\unit[195]{°C},
        %topheat=\unit[195]{°C},
        %gasstove=Level 2}},
    %portion = {\portion{16}},
    %calory={\unit[3]{kJ}},
    source = {Baba}
]
{Djuvic}
    
    \graph
    {% pictures
        small=pic/glass,     % small picture
        big=pic/ingredients  % big picture
    }
    
    \introduction{
		Pronounced djew-ich
    }
    
    \ingredients{%
        1 1/2 lbs & Pork \\
        1 1/2 cups & Rice \\
        & Hot Peppers \\
        & Bell Peppers \\
        4 & Large Onions \\
        3 & Medium Potatoes \\
        & Salt \\
        & Pepper \\
        & Accent \\
        4 cloves & Garlic \\
        1 large can & Tomatoes \\
    }
    
    \preparation{%
        \step Brown seasoned pork.  Remove pork from pan and add onions, garlic, peppers, and rice with butter. Saute.
        \step Par cook potatoes in water in microwave.  Add pork and potatoes to onions and rice.  Add tomatoes.  Cover with water and leave on stove until it boils.
        \step Pour olive oil on top and bake for at least one hour.  If it needs more water poke the rice and pour water over it.  Do not stir.
        \step Cook uncovered for the final 10 minutes.
    }
    
    %\suggestion[Finishing]
    %
    
    %\suggestion
    
    \hint{%
        Can be make with pork butt, country ribs, pork chops, etc.  Can also replace canned tomatoes with 2 fresh tomatoes.
    }
    
\end{recipe}