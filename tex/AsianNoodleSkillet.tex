%Complete recipe example
\begin{recipe}
[% 
    preparationtime = {\unit[1]{h}},
    %bakingtime={\unit[12-15]{min}},
    %bakingtemperature={\protect\bakingtemperature{
        %fanoven=\unit[375]{°F}}},
        %topbottomheat=\unit[195]{°C},
        %topheat=\unit[195]{°C},
        %gasstove=Level 2}},
    %portion = {\portion{32 scones}},
    %calory={\unit[3]{kJ}},
    source = {Pampered Chef}
]
{Asian Noodle Skillet}
    
    \graph
    {% pictures
        small=pic/glass,     % small picture
        big=pic/ingredients  % big picture
    }
    
    %\introduction{

    %}
    
    \ingredients{%
        2 & Medium Carrots, peeled \\
	1 & Red Bell Pepper \\
	5 - 6 & Green Onions with tops \\
	2 lbs & Pork Tenderloins \\
	2 Tbls & Toasted Sesame Oil \\
	2 Tbls & Asian Seasoning Mix \\
	4 pkg & Oriental-flavor Ramen Noodles \\
	1 Tbls & Vegetable Oil \\
	4 cups & Water \\
    }
    
    \preparation{%
        \step Cut vegetables and slice pork into thin slices.
	\step Combine pork, sesame oil, and seasoning mix and two of the ramen seasoning packets.
	\step Add vegetable oil to skillet; heat over medium-high heat for 1-3 minutes.  Add half the pork.  Cokk and stir 2-3 minutes or until browned.  Remove and keep warm while cooking the remaining pork.
	\step Add carrots and bell pepper to skillet.  Cook 1-2 minutes or until tender crisp.  Add water and remaining ramen seasoning packets.  Add ramen noodles and white parts of green onions to skillet.  Cover; bring to a boil and cook 4-5 minutes or until noodles are softened.  Add pork; stir to break apart noodles.  Remove from heat and let stand covered 3-4 minutes.
	\step Top with green onion tops.
    }
    
    %\suggestion[Finishing]
    %
    
    \suggestion{%
	Pork can be swapped for chicken, if desired.
    }
    
    %\hint{%
   %     Enjoy typesetting recipes with {\textbf{\Large\LaTeX}} and {\textbf{\Large xcookybooky!}}
  %  }
    
\end{recipe}
