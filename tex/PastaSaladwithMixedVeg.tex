%Complete recipe example
\begin{recipe}
[% 
    preparationtime = {\unit[1/2]{h}},
    %bakingtime={\unit[12-15]{min}},
    %bakingtemperature={\protect\bakingtemperature{
        %fanoven=\unit[375]{°F}}},
        %topbottomheat=\unit[195]{°C},
        %topheat=\unit[195]{°C},
        %gasstove=Level 2}},
    portion = {\portion{8}},
    calory={\unit[226]{kJ}},
    source = {Greg/Mayo Clinic}
]
{Pasta Salad with Mixed Vegetables}
    
    \graph
    {% pictures
        small=pic/glass,     % small picture
        big=pic/ingredients  % big picture
    }
    
    %\introduction{

    %}
    
    \ingredients{%
        12 oz & Farfalle \\
	1 Tbls & Olive Oil \\
	1/4 cup & Chicken Broth \\
	1 clove & Garlic, Minced \\
	2 & Medium Onions, Chopped \\
	1 28oz can & Diced Tomatoes in Juice \\
	1 lb & Mushrooms, Sliced \\
	1 & Red Bell Pepper, Sliced \\
	1 & Green Bell Pepper, Sliced \\
	2 & Medium Zucchini, shredded \\
	1/2 tsp & Basil \\
	1/2 tsp & Oregano \\
	8 & Romaine Lettuce Leaves \\
    }
    
    \preparation{%
        \step Fille a large pot 3/4 full with water and bring to a boil.  Add the pasta and cook until al dente.  Drain pasta thoroughly.  Place pasta in a large serving bowl.  Add olive oil, toss, and set aside.
	\step In large skillet, heat the chicken broth over medium heat.  Add the garlic, onions, and tomatoes.  Saute until the onions are transparent, about 5 minutes.  Stir in the basil and oregano.
	\step Add the vegetable mixutre to the pasta.  Toss to mix evenly.  Cover and refrigerate until well chilled.
	\step Place lettuce leaves on the individual plates and top with pasta salad to serve.
    }
    
    %\suggestion[Finishing]
    %
    
    %\suggestion
    
    %\hint{%
   %     Enjoy typesetting recipes with {\textbf{\Large\LaTeX}} and {\textbf{\Large xcookybooky!}}
  %  }
    
\end{recipe}
