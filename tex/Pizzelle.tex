% Complete recipe example
\begin{recipe}
[% 
    preparationtime = {\unit[1/2]{h}},
    %bakingtime={\unit[25-35]{min}},
    %bakingtemperature={\protect\bakingtemperature{
        %fanoven=\unit[350]{°F}}},
        %topbottomheat=\unit[195]{°C},
        %topheat=\unit[195]{°C},
        %gasstove=Level 2}},
    %portion = {\portion{16}},
    %calory={\unit[3]{kJ}},
    %source = {Somebody you used know \url{www.}}
]
{Pizzelles}
    
    \graph
    {% pictures
        small=pic/glass,     % small picture
        big=pic/ingredients  % big picture
    }
    
    %\introduction
    
    \ingredients{%
        6 & Eggs \\
        3 1/2 cups & Flour\\
        1 1/2 cups & Sugar\\
        1 cup & Butter (melted)\\
        4 tsp & Baking Powder\\
        2 Tbls & Vanilla (or anise)\\
    }
    
    \preparation{%
        \step Beat eggs adding sugar gradually.  Beat until smooth.
        \step Add cooled, melted butter and vanilla.
        \step Sift dry ingredients and add to egg mixture.
        \step Use pizzelle maker to bake them.
    }
    
    %\suggestion[Finishing]
    %
    
    \suggestion{
Try rolling some cookies around a small dowel while still hot and fill with ladylock filling.
    }
    
    \hint{
Pizzelles are delicious in many different varieties.  For \textbf{anise} flavored, replace part of the vanilla with one small bottle of anise oil (not extract).  For \textbf{chocolate} pizzelles, follow the recipe for vanilla pizzelles and add 1/2 cup cocoa, an additional 1/2 cup of sugar, and an additional half teaspoon baking powder.
    }
    
\end{recipe}